\documentclass[dvipdfmx]{beamer}
\usepackage[orientation=landscape,size=a0,scale=1.5]{beamerposter}%縦 : orientation=portrait, 横 : orientation=landscape
\usepackage{here, amsmath, latexsym, amssymb, bm, ascmac, mathtools, multicol, tcolorbox}

%カラーテーマの選択(省略可)
\usecolortheme{orchid}
%フォントテーマの選択(省略可)
\usefonttheme{professionalfonts}
%フレーム内のテーマの選択(省略可)
\useinnertheme{circles}
%ナビゲーションバー非表示
\setbeamertemplate{navigation symbols}{}
%既定をゴシック体に
\renewcommand{\kanjifamilydefault}{\gtdefault}
%itemize
\setbeamertemplate{itemize item}{\small\raise0.5pt\hbox{$\bullet$}}
\setbeamertemplate{itemize subitem}{\tiny\raise1.5pt\hbox{$\blacktriangleright$}}
\setbeamertemplate{itemize subsubitem}{\tiny\raise1.5pt\hbox{$\bigstar$}}
% color
\newcommand{\red}[1]{\textcolor{red}{#1}}
\newcommand{\green}[1]{\textcolor{green!40!black}{#1}}
\newcommand{\blue}[1]{\textcolor{blue!80!black}{#1}}
%head
\setbeamertemplate{headline}{
	\begin{center}
		\structure{
			\vskip3ex
			\rule{0.98\linewidth}{4mm}
			\vskip3ex
			\usebeamercolor{title in headline}{\textbf{\LARGE{\inserttitle}}\\[2.5ex]}
			\usebeamercolor{author in headline}{\large{\insertauthor}\\[1.2ex]}
			\usebeamercolor{institute in headline}{\large{\insertinstitute}}
			\vskip3ex
			\rule{0.98\linewidth}{4mm}
		}
	\end{center}
}
%foot
\setbeamertemplate{footline}{
	\begin{center}
		\structure{
			\rule{0.98\linewidth}{2mm}
			\vskip4ex
			\begin{columns}[T]
				\begin{column}{0.65\paperwidth}
				\end{column}
				\begin{column}{0.35\paperwidth}
					\usebeamercolor{conference in headline}{\large{\foot}}
				\end{column}
			\end{columns}
			\vskip3ex
		}
	\end{center}
}
\newtcolorbox{mybox}[1]
{
	title=#1, 
	toptitle=2mm, bottomtitle=2mm, 
	colframe=structure,boxrule=5pt,
	coltitle=white, colbacktitle=structure,
	colback=white, fonttitle=\bfseries\Large,
	top=5mm, bottom=5mm, left=5mm, right=5mm, %内部余白調整
	enlarge top by=1mm, enlarge bottom by=1mm, %外部余白調整
}

\title{ポスターのサンプル}
\author{著者名}
\institute{所属機関名}
\newcommand{\foot}{最下部に書きたいこと}

\begin{document}

\begin{frame}
\begin{columns}[t]
	\begin{column}{0.48\linewidth}
		\begin{mybox}{はじめに}
			あいうえお
		\end{mybox}
		\begin{mybox}{次に}
			\[
			\bm{x}^\top\bm{y}=1
			\]
		\end{mybox}
		\begin{mybox}{Test}
			Test
			\begin{itemize}
				\item a
				\begin{itemize}
					\item b
					\begin{itemize}
						\item c
					\end{itemize}
				\end{itemize}
			\end{itemize}
		\end{mybox}
	\end{column}
	\begin{column}{0.48\linewidth}
		\begin{mybox}{囲み系}
			\begin{block}{a}
				a
			\end{block}
			\begin{alertblock}{a}
				a
			\end{alertblock}
			\begin{exampleblock}{a}
				a
			\end{exampleblock}
			\begin{itembox}{a}
			a
			\end{itembox}
			\begin{tcolorbox}[colframe=green,
			colback=green!10!white,
			colbacktitle=green!40!white,
			coltitle=black, fonttitle=\bfseries,
			title=a]
			a
			\end{tcolorbox}
		\end{mybox}
	\end{column}
\end{columns}
\end{frame}

\end{document}
